
\section{Mapping subpatterns and inner}
In BETA a pattern can be a subpattern of another pattern just as a
class can be a subclass of another class. A subpattern can be mapped
directly to a subclass in CJ.

The inner-mechanism of BETA used for combination of the do-parts of a
pattern and its superpattern cannot be directly mapped into CJ.  In
the following example, an inner has been added to the {\tt withdraw}
pattern of {\tt Account}. In addition an {\tt owner} attribute and a
subpattern {\tt depositWithNotify} of {\tt withdraw} has been added.
 
\note{A better example is perhaps needed: with code before and after inner }
{\tt\small\begin{verbatim}
   Account:
      (# balance: @integer;
         owner: ^Person;
         deposit: 
          (# amount: @integer 
          enter amount 
          do balance + amount -> balance; 
             inner deposit
          #);

         depositWithNotify: withdraw
           (# 
           do amount -> owner.notify
           #)
      #)
\end{verbatim}
}

Execution of {\tt 60 -> myAccount.depositWithNotify} implies that the
do-part of {\tt deposit} is executed followed by an execution of the
do-part of {\tt depositWithNotify}.  This is captured by the following
mapping into CJ:

{\tt\small\begin{verbatim}
   class deposit extends Object {
      ...
      void do() {
         balance = balance + amount;
         do_1(); // inner deposit
      }
      protected void do_1()
   }
   class depositWithNotify extends deposit {
      void do_1() { 
         owner.notify(amount);
      }
   }
\end{verbatim}
}

In general there may be an arbitrary number of subpatterns and inners
and there may be statements to execute before and after an inner. This
is illustrated in the following example:

{\tt\small\begin{verbatim}
    A: (# do X1; inner; Y1 #);
    AA: A(# do X2; inner; Y2 #);
    AAA: AA(# do X3; inner; Y3 #);
\end{verbatim}
}

which will be mapped into:

{\tt\small\begin{verbatim}
   class A extends Object {
      void do() { X1(); do_1(); Y1(); };
      protected void do_1();
   }
   class AA extends A {
      protected void do_1() { X2(); do_2(); Y2(); };
      protected void do_2();
   }
   class AAA extends AA {
      protected void do_2() { X3(); do_3(); Y3(); };
      protected void do_3();
   }
\end{verbatim}
}

Consider instances:
{\tt\small\begin{verbatim}
   A a = new A(); AA aa = new AA(); AAA aaa = new AAA(); 
\end{verbatim}
}

\note{text is missing for the next example}
Execution of the do-method of theses objects takes places as follows
{\tt\small\begin{verbatim}
   a.do();   A.do(); X1(); ...
\end{verbatim}
}

As can be seen, inner is the empty action if there are no subpatterns
of a given pattern.

\note{Example of a call should be shown somewhere}
