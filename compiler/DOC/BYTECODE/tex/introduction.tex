\section{Introduction}
The overall goal of the project presented in this paper is language
interoperability between object-oriented languages. From a language
design/theoretical point-of-view issues concerning language
interoperability are interesting since they highlight essential
similarities and differences between languages. From an implementation
point of view, design and implementation of execution platforms
(virtual machine, common run-times) supporting a wide range of
languages are interesting.  From a practical point-of-view it is
interesting to be able to reuse libraries and frameworks between
across language borders. 
\note{Interesting for the following reasons ...}

The work reported here describes an exercise in implementing the BETA
language on Microsoft's .NET platform and the Java virtual machine
platform.  BETA is an object-oriented programming language and
implementations in the form off native compilers exists for a number
of platforms just as a powerful IDE -- the Mj�lner System -- is
available, see www.mjolner.dk.  For BETA all three issues mentioned
above are interesting.

At the language design level it is interesting to explore to what
extent languages are similar and just appear different due to
syntactic issues.  It is interesting to explore to what extent
mechanism found in one language can be simulated or abstracted in
another language. And finally it is interesting to find out to what
extent mechanisms differ fundamentally between languages.  For BETA
the the project has contributed to achieve a better understanding of
the parts of BETA that differ from other OO languages and it has
contributed with knowledge about which primitives and abstraction
mechanisms should be part of a language covering BETA and say the Java
and \CS\ family of languages.

Virtual machines with just-in-time compilers are becoming more and
more common as the basic technology for implementing object-oriented
languages. For some years Java from Sun Micro systems has been the
dominant technology in industry, but with Microsoft's .NET platform,
an alternative has arrived.  Both platforms are based on virtual
machines with a bytecode instruction set and type information that
makes it possible to verify the safety of a given program. The Java
platform has been designed solely for supporting the Java language
although there exist implementations for other languages. The
.NET-platform on the other hand was designed to support language
interoperability. On .NET it is e.g.\ possible to use a class written
in one language and make a subclass of it in another language. The
\CS\ language has been designed for the .NET-platform and is in many
ways similar to the Java language.

The porting of BETA to .NET and JVM has the following goals:
\begin{itemize}
\item To get experience with implementing a language like BETA on
  these platforms.  BETA differs in many respects from Java and \CS\ 
  and since the Java- and .NET platforms are both designed for
  implementing Java- and \CS-like languages, there may be parts of
  BETA that can not be easily implemented.
  
\item To be able to evaluate and compare the two platforms.  It is
  interesting to investigate whether there are significant differences
  between the two platforms.
  
\item To test language interoperability on these platforms.  One goal
  is to find out if language interoperability does work as promised by
  .NET. Another goal is to find out if language interoperability works
  with BETA and the other .NET languages. Since BETA differs
  significantly from e.g.\ \CS\ it is not obvious whether it is useful
  in practice to use a BETA pattern (class) in \CS\ or Visual Basic.
  Another goal is to experiment with language
  interoperability on the Java platform.
\item To offer a language BETA that can be used to write applications
  for both platforms.  If the porting of BETA to both
  platforms succeed, it will be possible to use BETA for implementing
  applications that can run on both platforms.
\end{itemize}

In \cite{eclipsepaper} work on integrating BETA with the Eclipse
integrated Development Environment (IDE) and Visual Studio has been
reported. This is another example of language interoperability.
