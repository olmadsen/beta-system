\appendix{Appendix -- Coroutines} In this appendix, the classes {\tt
Component} and {\tt Runner} are shown with the code for handling the
thread-pool.

{\tt\small\begin{verbatim}
   class Component {
      static Component current;     // The current executing Component
      private Component caller;     // The calling Component
      
      private static Runner firstFree;

      private Runner myRunner;      // the associated Runner
      private Coroutine body;       // The actual Coroutine
      private boolean isTerminated; // True if terminated
      
      public Component(Coroutine b) {
         body = b;
         caller = this;
         if (firstFree == null) {
            myRunner = new Runner();
         } else { // get at Runner from the thread-pool
            myRunner = firstFree;
            firstFree = firstFree.next;
            myRunner.myComponent = this;
         }
      }


      public void swap() {
         Component  old_current = current;
         current = caller;
         caller = old_current;

         synchronized(old_current.myRunner) {
            current.myRunner.go();
            if (old_current.isTerminated) {
               return;
            } else {
              try{ old_current.myRunner.wait(); }
              catch (InterruptedException e) {}
            }
         }         
      }
   }
\end{verbatim}
}

{\tt\small\begin{verbatim}
   class Runner extends Thread {
      Component myComponent;
      Runner next;'

      Runner(Component C) { 
         myComponent = C; 
         setDaemon(true) 
      }

      public void run() {
         while (true) {
            myComponent.body.Do();
            myComponent.isTerminated = true;
            myComponent.swap();

            // return this Runner to the thread-pool
            next = firstFree;
            firstFree = this;
            myComponent = null;

            // wait for a new Component to use this Runner
            synchronized (this) {
               try{ wait(); } catch (InterruptedException e){}
            }
         }
      }   
      public void go() {
        if !isAlive() { start(); }
        else {
           synchronized(this) {notify();};
        }
      }
   }
\end{verbatim}
}
