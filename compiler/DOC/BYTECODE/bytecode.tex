\documentclass{article}
%\documentclass{acm_proc_article-sp}
\def \CS{C\#}
\newcommand {\note}[1]{{\noindent\begin{quotation}\it Note! #1\end{quotation}}}

\usepackage{epsfig}
\usepackage[latin1]{inputenc}
\begin{document}
%\parindent 0pt
%\parskip 3mm

\title{Language Interoperability with BETA on .NET and Java Virtual Machines}
%\numberofauthors{1}                            
\author{ Ole Lehrmann Madsen \&\ Peter Andersen\\
%\alignauthor Jens Larsen\\
       Computer Science Department, Aarhus University\\
            �bogade 34, DK-8200 �rhus N, Denmark}
                              
\maketitle 

                         
\begin{abstract}
\note{Current version of how BETA is mapped into CLR/JVM}

\end{abstract}

\section{Introduction}
The overall goal of the project presented in this paper is language
interoperability between object-oriented languages. From a language
design/theoretical point-of-view issues concerning language
interoperability are interesting since they highlight essential
similarities and differences between languages. From an implementation
point of view, design and implementation of execution platforms
(virtual machine, common run-times) supporting a wide range of
languages are interesting.  From a practical point-of-view it is
interesting to be able to reuse libraries and frameworks between
across language borders.

The work reported here describes an exercise in implementing the BETA
language on Microsoft's .NET platform and the Java virtual machine
platform.  BETA is an object-oriented programming language and
implementations in the form off native compilers exists for a number
of platforms just as a powerful IDE -- the Mj�lner System -- is
available, see www.mjolner.dk.  For BETA all three issues mentioned
above are interesting.

At the language design level it is interesting to explore to what
extent languages are similar and just appear different due to
syntactic issues.  It is interesting to explore to what extent
mechanism found in one language can be simulated or abstracted in
another language. And finally it is interesting to find out to what
extent mechanisms differ fundamentally between languages.  For BETA
the the project has contributed to achieve a better understanding of
the parts of BETA that differ from other OO languages and it has
contributed with knowledge about which primitives and abstraction
mechanisms should be part of a language covering BETA and say the Java
and \CS\ family of languages.

Virtual machines with just-in-time compilers are becoming more and
more common as the basic technology for implementing object-oriented
languages. For some years Java from Sun Micro systems has been the
dominant technology in industry, but with Microsoft's .NET platform,
an alternative has arrived.  Both platforms are based on virtual
machines with a bytecode instruction set and type information that
makes it possible to verify the safety of a given program. The Java
platform has been designed solely for supporting the Java language
although there exist implementations for other languages. The
.NET-platform on the other hand was designed to support language
interoperability. On .NET it is e.g.\ possible to use a class written
in one language and make a subclass of it in another language. The
\CS\ language has been designed for the .NET-platform and is in many
ways similar to the Java language.

The porting of BETA to .NET and JVM has the following goals:
\begin{itemize}
\item To get experience with implementing a language like BETA on
  these platforms.  BETA differs in many respects from Java and \CS\ 
  and since the Java- and .NET platforms are both designed for
  implementing Java- and \CS-like languages, there may be parts of
  BETA that can not be easily implemented.
  
\item To be able to evaluate and compare the two platforms.  It is
  interesting to investigate whether there are significant differences
  between the two platforms.
  
\item To test language interoperability on these platforms.  One goal
  is to find out if language interoperability does work as promised by
  .NET. Another goal is to find out if language interoperability works
  with BETA and the other .NET languages. Since BETA differs
  significantly from e.g.\ \CS\ it is not obvious whether it is useful
  in practice to use a BETA pattern (class) in \CS\ or Visual Basic.
  Another goal is to experiment with language
  interoperability on the Java platform.
\item To offer a language BETA that can be used to write applications
  for both platforms.  If the porting of BETA to both
  platforms succeed, it will be possible to use BETA for implementing
  applications that can run on both platforms.
\end{itemize}

In \cite{eclipsepaper} work on integrating BETA with the Eclipse
integrated Development Environment (IDE) and Visual Studio has been
reported. This is another example of language interoperability.

\section{Am immediate mapping of BETA to CLR/JVM}

There have been two issues regarding the mapping of BETA to CLR/JVM:
\begin{enumerate}
  
\item CLR/JVM are typed virtual machines designed to support Java for
  JVM and a number of languages like \CS, and VB for CLR. BETA is much
  more general than Java and \CS\ so one issue has been to find a
  mapping that works.

\item The other issue is language interoperabiliby as seen from the
  point-of-view of programmers. To obtain true language
  interoperability, is should be possible to use classes written in
  Java, \CS, etc. from BETA and vice versa. Given an implementation of
  BETA for CLR/JVM this should be possible. The main issue will be
  readability/understandability of doing this. Using Java and \CS\  from
  BETA will probably not pose problems, since a BETA programmer just
  have to understand the Java/\CS\ classes being used.
  
  The other way around may be more problematic. Since BETA is much
  more general than Java and \CS, the mapping of BETA patterns to
  CLR/JVM may without careful design be complicated and result in
  class-files that are difficult to understand at the level of Java and
  \CS.

\end{enumerate}

The mapping of BETA into CLR/JVM is described as as a mapping into a
common subset of \CS\ and Java. The subsequent mapping into CLR/JVM
bytecodes should then appear as straight forward.  That is a the
\CS/Java level the mappings to the two platforms are essentially
identical. There are of course differences when considering the
bytecode level but these are purely technical -- the logical structure
is identical.

The language model -- as we from now on call {\it CJ} -- behind Java
and \CS\ may be characterized in the following way:
\begin{itemize}
\item A program is a collection of classes where each
  class defines the structure of a set of objects.
\item A class defines
\begin{itemize}
\item A set of data-items -- instance variables belonging to instances
  of the class, and static variables belonging to the class.
\item A set of methods -- instance methods and static methods.
\end{itemize}
\item Classes may be textually nested: in Java real nesting in the
  form of block-structure as in Simula and BETA is supported whereas
  \CS\ only provides nesting as a scoping mechanism similar to C++.
\item Methods may be overloaded.
\item There is support for dynamic exceptions.
\item Concurrency is supported in the form of threads and a
  monitor-like construct.
\end{itemize}

The BETA language may be characterized in the following way:
\begin{itemize}
\item Class and method have been unified into the pattern mechanism.
\item Patterns can be arbitrarily nested supporting general
  block-structure.
\item
The INNER-mechanism is used to combine methods instead of super.
\item Genericity is supported by means of virtual patterns.
\item A pattern (method) invocation may return multiple values.
\item Coroutines and concurrency is based on semi-coroutines as known
  from Simula in the form of active objects. Basic synchronization
  between concurrent objects is supported by a semaphore pattern.
  These mechanisms are at a more basic level than threads and monitors
  in CJ, but form a basis for implementing higher-level concurrency
  abstractions like monitors and Ada-like rendezvous and schedulers.
\item BETA has no constructor mechanism.
\item Methods cannot be overloaded.
\item BETA is based on a static exception handling mechanism
  \cite{lindskov} and dynamic exceptions are only supported in an
  experimental version of the language.
\end{itemize}

Note that this paper is not an attempt to present the rationale behind
the generality of BETA compared to main stream languages like \CS\ and
Java.  The purpose is to present the issues in mapping BETA to
CLR/JVM. To do this we will use more or less useful examples to
illustrate the mapping. The rationale behind BETA has been presented
in a number of articles such as \cite{betabook,vclass}.


Consider the following example of a BETA pattern describing a
simple bank account:
{\tt\small\begin{verbatim}
   Account:
     (# balance: @integer;
        deposit: 
          (# amount: @integer 
          enter amount 
          do balance + amount -> balance 
          exit balance
          #);
        withdraw: 
          (# amount: @integer enter amount do balance - amount -> balance #);
     #);
\end{verbatim}
}
    
The {\tt Account} pattern has three attributes {\tt balance}, {\tt
  deposit} and {\tt withdraw}. The attribute {\tt balance} is an
instance variable holding the current value of the {\tt Account}.  The
attributes {\tt deposit} and {\tt withdraw} describe operations on the
account.  The operation {\tt deposit} inserts an amount on the account
by adding to {\tt balance} to the value of the enter-parameter {\tt
  amount} and returns (exit) the value of {\tt balance}.  The
operation {\tt withdraw} similarly withdraws an amount from the
account.
    
The following example shows an instance {\tt myAccount} of {\tt
  Account}, an instance {\tt X} of {\tt integer}; a call of {\tt
  myAccount.deposit} with enter arguments {\tt 120} and the resulting
exit-value being assigned to {\tt X}; and a call of {\tt
  myAccount.withdraw} with enter argument {\tt 50}.

{\tt\small\begin{verbatim}
   myAccount: @Account; X: @integer; 
   120 -> myAccount.deposit -> X;
   50 -> C.withdraw;
\end{verbatim}
}
    
Note that {\tt Account}, {\tt deposit} and {\tt withdraw} are all
examples of patterns. In this example {\tt Account} is used as a class
and {\tt deposit} and {\tt withdraw} as methods.  This use of the {\tt
  Account} pattern is illustrated by the following simple mapping to a
CJ class:

{\tt\small\begin{verbatim}
   class Account extends Object { 
      int balance;
      void deposit(int amount) { 
         balance = balance + amount; 
         return balance; 
      }
      int withdraw(int amount) { balance = balance - amount; }
   }
\end{verbatim}
}

The BETA declarations and invocations shown above then maps to the
following declarations and invocations in CJ:

{\tt\small\begin{verbatim}
   Account myAccount = new Account(); int X;
   X = myAccount.deposit(120);
   myAcount.withdraw(50);
\end{verbatim}
}
    
The above mapping shows a simple semantics of the BETA {\tt Account}
pattern.  To capture the full semantics of the {\tt Account} pattern a
more complex mapping is needed. The fact that e.g.\  {\tt deposit} is a
pattern means that it is possible to use {\tt deposit} as a class and
create instances of {\tt deposit} as in the following example:

{\tt\small\begin{verbatim}
   myAccount: @Account; 
   myDeposit: ^myAccount.deposit; 
   myAccount.deposit[] -> myDeposit[];
\end{verbatim}
}
    
The variable {\tt myDeposit} may refer to instances of {\tt
  myAccount.deposit}. The statement {\tt \&myAccount.deposit[] ->
  myDeposit[]} creates an instance of the pattern {\tt
  myAccount.deposit} and assigns its reference to {\tt myDeposit}.

Using {\tt myDeposit}, the instance variable {\tt amount} in {\tt
  myDeposit} may be assigned a value as in: 

{\tt\small\begin{verbatim}
   170 -> myDeposit.balance
\end{verbatim}
}
    
The pattern {\tt deposit} also defines a do-part, which is executed
when {\tt deposit} is used a a procedure. The do-part of {\tt
  myDeposit} may be executed directly by the following statement (the
exact semantics should be clear when we show the complete mapping of
the {\tt Account} pattern below):

{\tt\small\begin{verbatim}
   20 -> myDeposit -> X
\end{verbatim}
}

To obtain the full semantics of the {\tt deposit} pattern, it is
mapped into the following inner class\footnote{%
  The reader may observe that this use of an inner class will work in
  Java, but not in \CS. This issue is further discussed below.}  of 
{\tt Account}:

{\tt\small\begin{verbatim}
   class deposit extends Object {
      int amount;
      void enter(int a) { amount = a; }
      void do() { balance = balance + a; }
      int exit() { return balance; }
   }
\end{verbatim}
}

A BETA invocation
{\tt\small\begin{verbatim}
    50 -> myAccount.deposit -> X
\end{verbatim}
}
is then implemented as follows:
{\tt\small\begin{verbatim}
   deposit D = new deposit();
   D.enter(5);
   D.do();
   X = D.exit();
\end{verbatim}
}

To avoid generating the above code for each invocation, and {\tt
  deposit}-method is generated as part of class {\tt Account}:

{\tt\small\begin{verbatim}
   int deposit(int amount) {
       deposit D = new deposit();
       D.enter(50);
       D.do();
       return D.exit();
   }
\end{verbatim}
}

The BETA invocation {\tt 50 -> myAccount.deposit -> X} may now be
mapped into the corresponding CJ invocation {\tt X =
  myAccount.deposit(50)}.

\note{Drop new-method here}

As mentioned,it is also possible to generate instances of the patter
{\tt deposit}. To handle this a {\it new}-method of the following form
is generated as part of class {\tt Account}.

{\tt\small\begin{verbatim}
   deposit deposit() { return deposit(); }
\end{verbatim}
}

Instances of {\tt deposit} may now be generated by execution of {\tt
  myDeposit = myAccount.deposit()}.

The complete mapping of the {\tt Account} pattern is shown below:
{\tt\small\begin{verbatim}
   class Account extends Object {
     int balance = 0;
     int deposit(int amount) {
       deposit D = new deposit();
       D.enter(amount);
       D.do();
       return D.exit();
     }
     void withdraw(int amount) { ... };
     class deposit extends Object { ... };
     class withdraw extends Object { ... };
   }
\end{verbatim}
}

As can be seen, each inner pattern of {\tt Account} -- in this case
{\tt deposit} and {\tt withdraw} -- gives rise to a method and an
inner class.

The method is for using the pattern as a method -- for {\tt withdraw}
this is the method with the signature {\tt int withdraw(int amount)}.

The {\tt Account} example also shows how to use BETA patterns from
Java and \CS. A Java or \CS\ class may use the {\tt Account} class as
shown above and invoke the {\tt deposit}- and {\tt deposit}-methods
as shown in the following Java example:

{\tt\small\begin{verbatim}
   Account myAccount = new Account();
   deposit myDeposit = myAccount.new deposit()
   mydeposit.enter(170);
   mydeposit.do();
   int X = myDeposit.exit();
\end{verbatim}
}

In the following sections we will more systematically show how
BETA constructs are mapped into CJ.

\section{The general scheme for mapping a pattern}
\note{Consider skipping this part -- is very sketchy}

Consider a pattern with nested patterns:
{\tt\small\begin{verbatim}
   Foo:
      (# ...
         f1: (# ... enter N1 do ... exit X1 #);
         f2: (# ... enter N2 do ... exit X2 #);
         f3: (# ... enter N3 do ... exit X3 #);
      enter N
      do ...
      exit X
      #)
\end{verbatim}
}
\note{We still need to introduce the new-method}

Pattern {\tt Foo} is mapped into the following class:
{\tt\small\begin{verbatim}
   class Foo extends Object {
      ...
      sX1 f1(sN1) { ... } // call-method for f1
      f1 new$f1() { ... } // new-method for f1

      sX2 f2(sN2) { ... } // call-method for f2
      f2 new$f2() { ... } // new-method for f2

      sX3 f3(sN3) { ... } // call-method for f3
      f3 new$f3() { ... } // new-method for f2

      void enter(sN) { ... }
      void do() { ... }
      sX exit() { ... }

      class f1 extends Object { ... }
      class f2 extends Object { ... }
      class f3 extends Object { ... }
   }
\end{verbatim}
}

\noindent
where {\tt sX1}, {\tt sN1}, \ldots, {\tt sN}, and {\tt sX} are the
signatures corresponding to {\tt X1},{\tt N2}, \ldots {\tt N}, and {\tt
  X} respectively.

\section{Mapping nested patterns}
Consider the following structure of nested (inner) patterns:
{\tt\small\begin{verbatim}
   A: (# ...
         AA: (# ...
                AAA: (# ... #)
             #)
      #)
\end{verbatim}
}

This structure is mapped into the following structure of nested
patterns:

{\tt\small\begin{verbatim}
   class A extends Object { 
      class AA extends Object { 
         ...
         class AAA extends Object {
            ...
         }
      }
   }
\end{verbatim}
} 

As mentioned above, this works for Java but not for \CS.  In \CS\ 
instances of inner classes cannot refer to instance variables in the
enclosing object.  To handle this BETA and Java maintains a structural
reference to the enclosing object of an inner class.  For \CS, we have
to generate this reference as part of the mapping. The above nested
BETA patterns are thus mapped into the following \CS\ structure:

{\tt\small\begin{verbatim}
   class A extends Object { 
      class AA extends Object { 
         A origin;
         AA(A org) { origin = org;}
         ...
         class AAA extends Object {
            AA origin; 
            AAA(AA org) { origin = org; }
            ...
         }
      }
   }
\end{verbatim}
} 

Each inner class has an explicit reference, {\tt origin} to its
enclosing object. The origin reference is setup by a constructor that
has the origin as an argument.

\note{Figure with nested A, AA, and AAA instances?}

The {\tt Account} example thus maps into the following classes for
\CS:

{\tt\small\begin{verbatim}
   class Account extends Object {
      int balance = 0;
      int deposit(int amount) {
          deposit D = new deposit(this);
          D.enter(amount);
          D.do();
          return D.exit();
      }
      void withdraw(int amount) { ... };
      class deposit extends Object { 
         origin Account;
         int amount;
         deposit(origin org) { origin = org; }
         void enter(int a) { amount = a; }
         void do() { origin.balance = origin.balance + amount; }
         int exit() { return origin.balance; }
      }
      class withdraw extends Object { ... };
}
\end{verbatim}
}

\note{Some explanation ...}

An inspection of the class-files and generated by Java (and \CS?)
reveals that inner classes are flattened as outer classes and thus all
appear at the same level of block structure.  To handle this a name
mangling scheme is used by Java (and \CS?) to ensure uniqueness of
class names. The above BETA patterns then at the class-file level are
mapped into the following flat structure of classes:

{\tt\small\begin{verbatim}
   class A extends Object { ... }
   class A$AA extends Object { 
      A origin;
      A$AA(A org) { origin = org; }
      ...
   }
   class A$AA$AAA extends Object {
      A$AA orgin;
      A$AA$AAA(A$AA org) { origin org; }
      ...
   }
\end{verbatim}
} 
  
The flattening, name mangling and explicit origin reference are only a
matter at the class-file level. At the user level -- Java and \CS\ 
programmers using BETA, the flattening does not show up.  Except that
\CS\ programmers have to deal with the origin reference.

\note{We should either make use of Java inner classes or have a remark
  about it. I.e. that we currently do not use them but plan to.}

\section{Mapping subpatterns and inner}
In BETA a pattern can be a subpattern of another pattern just as a
class can be a subclass of another class. A subpattern can be mapped
directly to a subclass in CJ.

The inner-mechanism of BETA used for combination of the do-parts of a
pattern and its superpattern cannot be directly mapped into CJ.  In
the following example, an inner has been added to the {\tt withdraw}
pattern of {\tt Account}. In addition an {\tt owner} attribute and a
subpattern {\tt depositWithNotify} of {\tt withdraw} has been added.
 
\note{A better example is perhaps needed: with code before and after inner }
{\tt\small\begin{verbatim}
   Account:
      (# balance: @integer;
         owner: ^Person;
         deposit: 
          (# amount: @integer 
          enter amount 
          do balance + amount -> balance; 
             inner deposit
          #);

         depositWithNotify: withdraw
           (# 
           do amount -> owner.notify
           #)
      #)
\end{verbatim}
}

Execution of {\tt 60 -> myAccount.depositWithNotify} implies that the
do-part of {\tt deposit} is executed followed by an execution of the
do-part of {\tt depositWithNotify}.  This is captured by the following
mapping into CJ:

{\tt\small\begin{verbatim}
   class deposit extends Object {
      ...
      void do() {
         balance = balance + amount;
         do_1(); // inner deposit
      }
      protected void do_1()
   }
   class depositWithNotify extends deposit {
      void do_1() { 
         owner.notify(amount);
      }
   }
\end{verbatim}
}

In general there may be an arbitrary number of subpatterns and inners
and there may be statements to execute before and after an inner. This
is illustrated in the following example:

{\tt\small\begin{verbatim}
    A: (# do X1; inner; Y1 #);
    AA: A(# do X2; inner; Y2 #);
    AAA: AA(# do X3; inner; Y3 #);
\end{verbatim}
}

which will be mapped into:

{\tt\small\begin{verbatim}
   class A extends Object {
      void do() { X1(); do_1(); Y1(); };
      protected void do_1();
   }
   class AA extends A {
      protected void do_1() { X2(); do_2(); Y2(); };
      protected void do_2();
   }
   class AAA extends AA {
      protected void do_2() { X3(); do_3(); Y3(); };
      protected void do_3();
   }
\end{verbatim}
}

Consider instances:
{\tt\small\begin{verbatim}
   A a = new A(); AA aa = new AA(); AAA aaa = new AAA(); 
\end{verbatim}
}

\note{text is missing for the next example}
Execution of the do-method of theses objects takes places as follows
{\tt\small\begin{verbatim}
   a.do();   A.do(); X1(); ...
\end{verbatim}
}

As can be seen, inner is the empty action if there are no subpatterns
of a given pattern.

\note{Example of a call should be shown somewhere}

\section{Mapping virtual patterns}

A BETA pattern may be virtual. A virtual pattern used as as procedure
corresponds to a (virtual) method in Java and \CS. A virtual pattern
used as a class corresponds to a class parameter of a generic class
(parameterized class) in Java -- there is no counterpart in \CS\ (Is
this correct?).

\note{More references to virtual class literature}

The following example shows the use of virtual patterns. The pattern
{\tt Buffer} is a generic buffer that may hold an instance of the
pattern {\tt elm}.  Pattern {\tt elm} is declared as a virtual pattern
attribute of {\tt Buffer} -- the {\tt <} in {\tt elm:< object}
indicates that {\tt elm} is a virtual attribute -- the {\tt object}
part of the declaration indicates that {\tt elm} may be extended in
subpatterns of {\tt Buffer} to subpatterns of {\tt object}. This means
that the type (qualification) of {\tt elm} in {\tt Buffer} is {\tt
  object} and that {\tt elm} can be used as if it is pattern {\tt
  object}.

The instance variable {\tt theElm}, the enter-argument {\tt E} of
{\tt insert} and the exit-value {\tt X} of {\tt get}
are all declared to be of type {\tt elm}, i.e. in {\tt
  Buffer}, {\tt object}.

Since {\tt object} is the most general superpattern in BETA, only few
properties of {\tt elm } is known in {\tt Buffer}, but in subpatterns
of {\tt Buffer}, {\tt elm} may be extended to any other pattern.

\note{display - currently ignored}

{\tt\small\begin{verbatim}
   Buffer:
     (# elm:< object;
        theElm: ^elm;
        put:
          (# E: ^elm
          enter E[] do E[] -> theElm[]
          #);
        get: 
          (# X: ^elm
          do theElm[] -> X[]
          exit E[]
          #);
        display:<
          (#
          do 'Buffer: ' -> screen.puttext; inner;
          #)
     #);
\end{verbatim}
}

Since {\tt elm} is qualified by {\tt object} arbitrary objects may be
inserted into a {\tt Buffer} as shown in the following example where
an {\tt Account} and a {\tt Person} are inserted into a {\tt
  Buffer}-object.

{\tt\small\begin{verbatim}
   (# B: @Buffer; 
      Joe: @Person;
      myAccount: @Account
      X: ^object;
   do myAccount[] -> B.put;
      Joe[] -> B.put;
      B.get -> X[];
   #)
\end{verbatim}
}

The statement {\tt B.get -> X[]} retrieves the element from the buffer
and assigns it to {\tt X}. Since {\tt get} returns a reference of type
{\tt elm} the type of {\tt X} also has to be {\tt object}

Also the type of en element retrieved from the buffer by {\tt get}
is only know to be of type {\tt object}.

In pattern {\tt AccountBuffer} -- a subpattern of {\tt Buffer} -- {\tt
  elm} is extended to be the pattern {\tt Account} as specified by the
declaration {\tt elm::< Account}. This means that inside {\tt
  AccountBuffer}, we know that the instance variable {\tt theElm} is
at least of type {\tt Account}. This is also the case for the
enter-argument {\tt E} of put and the exit-value {\tt X} of get.


%This fact is used i the {\tt display}
%method where {\tt theElm} is used is written as a text to the screen.

{\tt\small\begin{verbatim}
   AccountBuffer: Buffer
     (# elm::< Account
        display::< 
          (#
          do theElm.balance -> putint;
          #)
     #);
\end{verbatim}
}

The next example shows a use {\tt AccountBuffer}
where only instances of {\tt Account} or subpatterns of {\tt Account}
may be inserted:

{\tt\small\begin{verbatim}
   (# Joe: @Person;
      myAccount: @Account
      Q: @AccountBuffer; 
      S: ^Account
   do myAccount[] -> B.put;
      Joe[] -> B.put;
      B.get -> X[];
      myAccount[] -> Q.put;
      Q.get -> S[];
   #)
\end{verbatim}
}

In addition the type of elements retrieved from {\tt Q} is known to be
of type {\tt Account}

\subsection{First version of mapping}
\note{Better heading needed} The pattern {\tt Buffer} is mapped into
the following Java class.  The operations {\tt put} and {\tt get} are
mapped into a method and an inner class following the general mapping
for nested patterns. 

The special thing here is the treatment of the virtual pattern {\tt
  elm}. Since {\tt elm} is declared as a virtual {\tt object}, the
type of {\tt theElm}, the argument {\tt a} of {\tt put} and the
argument {\tt a} of the enter-method of class {\tt put} all become
{\tt Object}.\footnote{There is a subtle difference between {\tt
    object} in BETA and {\tt Object} in Java that we will ignore in
  this paper}

{\tt\small\begin{verbatim}
   class Buffer extends Object {
     Object theElm;

     void put(Object a) {
        put I = new put();
        I.enter(a);
        I.do         
       }
     Object get() { 
        get G = new get();
        G.do();
        return G.exit();
     }

     class put {
        Object E;
        void enter(Object a) { E = a; }
        void do() { theElm = E; }
     }
     class get { 
        Object X;
        void do() { X = theElm; }
        Object exit() { return X; }
     }
   }
\end{verbatim}
}

The BETA example above of using {\rm Buffer} is straight forward to
map into the following CJ program:

{\tt\small\begin{verbatim}
   Account B;
   Person Joe = new Person();
   Object X;
   Account myAccount = new Account();
   Account S;
   B.put(myAccount);
   B.put(Joe);
   X = B.get();
\end{verbatim}
}


The pattern {\tt AccountBuffer} is mapped into the following Java class:

{\tt\small\begin{verbatim}
   class AccountBuffer extends Buffer {
   }
\end{verbatim}
}

As can bee seen, the body of the class is empty\footnote%
{This is not entirely true since the current version of JBeta
  generates dummy-methods corresponding to {\tt elm} in {\tt Buffer}}
and {\tt AccountBuffer}, the reason being that the body of the BETA
{\tt AccountBuffer} only extends the virtual pattern {\tt elm} to be
at least {\tt Account}. This extension cannot be expressed in Java.
Instead we have to make use of casts at various places.  The BETA
example of using {\tt AccountBuffer} thus maps into:

{\tt\small\begin{verbatim}
   Person Joe = new Person();
   Account myAccount = new Account();
   AccountBuffer Q;
   Account S;

   Q.put(myAccount);
   S = (Account) Q.get();    // cast
\end{verbatim}
}

Note the statement {\tt S = (Account) Q.get();} where the value
returned by {\tt Q.get()} must be casted to {\tt Account}. This cast
is superfluous since the BETA compiler knows that it will not be
needed. This can as mentioned not be expressed in Java.

\note{Covariance should be mentioned}

\subsection{Mapping {\tt  display} into CJ}
The mapping of the virtual pattern {\tt display} follow the general
scheme of mapping a nested pattern: a method and an inner class is
generatated as shown below:


{\tt\small\begin{verbatim}
   class Buffer extends Object {
     ...

     void display() {
        display D = new display();
        D.do();
      }

     class display extends Object { 
        void do() {
          System.out.println(``Buffer: ``);
          do_1();
        }
        void do_1();
     }
   }
\end{verbatim}
}
 
In {\tt AccountBuffer}, the extension of {\tt display} gives rise to a
redefinition of the {\tt display}-method and -class:

{\tt\small\begin{verbatim}
   class AccountBuffer extends Buffer {
     ...

     void display() {
        display D = new display();
        D.do();
      }

     class display extends Buffer.display { 
       void do_1() { System.out.println(theElm.balance); }
     }
   }
\end{verbatim}
}

As can be seen a new {\tt display}-class is defined in {\tt
  AccountBuffer} and this class is a subclass of the {\tt
  display}-class from {\tt Buffer}.  The {\tt display}-method
generates an instance of the {\tt display}-class from {\tt
  AccountBuffer}. In

{\tt\small\begin{verbatim}
    Buffer B = new Buffer();
    B.display;
    Buffer A = new AccountBuffer();
    A.display()
\end{verbatim}
}

{\tt B.display()} executes {\tt display} as defined in {\tt Buffer}
whereas {\tt A.display} executes {\tt display} as defined in {\tt
  AccountBuffer} according to the BETA semantics of method
combination using inner.

\subsection{Mapping nested virtual classes}
A virtual pattern may also be a nested pattern as shown in the next example:

{\tt\small\begin{verbatim}
Structure:
  (# root: @Node;
     Node:< 
       (# label: @integer;
          add:< (# N: ^Node enter N[] do inner #);
       #);
     add:
       (# V: @integer; N: ^Node
       enter V
       do &Node[] -> N[];
          V -> N.label;
          N[] -> root.add
       #) 
  #);
\end{verbatim}
}

An instance of pattern {\tt Structure} has three attributes: an
instance variables {\tt root} of type {\tt Node}, a nested virtual
pattern {\tt Node}, and an operation {\tt add} for adding a {\tt Node}
to the {\tt structure}. A {\tt Node} has two attributes: an instance
variable {\tt label} and an operation {\tt add}.

Since {\tt node} is a nested virtual pattern, {\tt Node} may be
extended in subpatterns of {\tt Structure}. Instances of {\tt Node}
will then be instances of the extended pattern. In {\tt Structure} two
instances of {\tt Node} are created: one by {\tt root: @Node} and one
in {\tt add} by the statement {\tt \&Node[] -> N[]}. These
instance will be of a possibly extension of {\tt Node}.

In pattern {\tt List}, {\tt Node} is extended with an instance
variable {\tt succ} and an extension of {\tt add}.

{\tt\small\begin{verbatim}
List: Structure 
  (# 
     Node::< 
       (# succ: ^Node;
          add::< (# ... #)
       #)
  #);
\end{verbatim}
}

If {\tt L} is a {\tt List} then {\tt L.root} and the instances created
by {\tt L.add} below are instance of the extended {\tt Node} pattern.

The extension of the virtual {\tt add} is supposed to append the
enter-argument {\tt N} to the list defined by {\tt succ}.

{\tt\small\begin{verbatim}
(# L: @List;
do (for i: 10 repeat i -> L.add for);
#)
\end{verbatim}
}

The next example shows another example of an extension of {\tt Node}
in a subpattern {\tt BinTree} of {\tt Structure}:

{\tt\small\begin{verbatim}
BinTree: Structure
  (#
     Node::< 
       (# left,right: ^Node;
          addRight: @boolean;
          add::< (# ... #);
       #)
  #)
\end{verbatim}
}

Pattern {\tt Node} is extended with instance variables {\tt left},
{\tt right} and {\tt addRight}, the latter in the extension of {\tt
  add} to alternate between adding {\tt N} to the left- or right part
of the binary tree.

The example shows how to add elements to a {\tt BinTree}.

{\tt\small\begin{verbatim}
(# T: @BinTree;
do (for i: 10 repeat i -> T.add for);
#)
\end{verbatim}
}

\subsubsection{Mapping of nested virtuals into CJ}

Pattern {\tt Structure} is mapped into the CJ class shown below. The
parts relating to the nested virtual {\tt Node}-pattern is the
method{\tt Node new\$Node()} and the inner {\tt class Node}. For
simplicity the {\tt add}-methods of {\tt Structure} are shown as
CJ-methods although they are really mapped into a method and a class:


{\tt\small\begin{verbatim}
   class Structure {
       Node root = new$Node();

       Node new$Node() { return new Node(); }

       void add(int V) {
          Node N = new$Node();
          N.label = V;
          root.add(N);
       }    
       
      class Node {
         int label;
         void add(Node N){ ... };
      }
   }
\end{verbatim}
}

Pattern {\tt List} is mapped into the following CJ class:

{\tt\small\begin{verbatim}
   class List extends Structure {

       Structure.Node new$Node() { return new Node(); }

       class Node extends Structure.Node {
          Structure.Node succ;
          void add(Structure.Node N) { ... }

       }
   }
\end{verbatim}
} 
  
\note{We use {\tt Structure.Node} above, to ensure that {\tt
    new\$Node()} is overwritten. Could we use {\tt Node} instead?
  Return type is not used to distinguish signatures?}

Here it should be noticed that a an inner class {\tt Node}
corresponding to the extension of {\tt Node} in {\tt List} is defined.
This class is a subclass of the {\tt Node} class (denoted {\tt
  Structure.Node} defined in {\tt Structure}.

Also, the method with signature {\tt Structure.Node new\$Node() } is a
redefinition of the corresponding method in {\tt Structure}. The
redefinition ensures that an instance of the {\tt List.Node} is
returned in accordance with the semantics of virtual classes.

Finally we consider the use of {\tt List}:

{\tt\small\begin{verbatim}
   List L = new List();
   for (i=0; i<=10; i++) L.add(i); 
\end{verbatim}
}

\note{Need external instances L.Node, T.Node}

\subsection{Summing up}

For a pattern: a call- and a new-method.

Also the case for a virtual pattern

For a non-nested (global), no inner class

For a nested virtual: both

Special dummy/empty cases that may be avoided

Extra casts are needed

Note, however, that how close CJ is to support virtual patterns.
Mainly a matter of the new-method and avoiding the casts.
We may simulate virtual classes by the above means.


\section{Expressions and statements}
Expressions and statemenst are in most cases straight forward to map intp CJ.

{\bf Basic types}. BETA has basic types corresponding to 8, 16, and 32
bits signed and unsigned integers, 8, and 16-bit characters, and 64
bit floating point numbers. (, 16, and 32-bits signed number are
directly supported by CJ - for the unsigned version a little extra
effort is needed.

{\bf Binary and unary operators}. All of the binary and unary
operators of BETA maps direcly to corresponding operators in CJ.

{\bf Multiple return values}. A BETA pattern may return a list of
values. This can not be directly implemented in JVM/CLR -- instead the
list of return values are stored in instance variables and a reference
to the method-invocation is returned and used subsequently to retrieve
the return list.

{\bf Statements}. BETA's if-, and for-statements also maps easily into CJ.

\subsection{Leave and restart}
In BETA is possible to exit a nested scope of method invocations:

{\tt\small\begin{verbatim}
   Foo:
     (#
     do ...;
        L: (# Bar: 
                (# 
                do ...; leave L; 
                   ...; restart L; ...
                #);
           do ...; Bar; ...
           #);
        Lx: ...
     #)
\end{verbatim}
}
  
Execution of {\tt leave L} implies that execution continues at the
point of {\tt Lx} --- equivalent to {\tt goto Lx}. Execution of {\tt
  restart L} implies that execution continues at {\tt L} ---
equivalent to {\tt goto L}. Execution of {\tt leave}/{\tt restart} may
take place within arbitrarely nested pattern invocations\footnote{%
  The patterns must be declared in the scope of {\tt L}, i.e. within
  the object labelled by {\tt L}} --- in the example shown within the
pattern {\tt Bar}. 

Notice that this involves {\em stack unwinding}, since the stack
frame for {\tt Bar} must be removed, before execution continues at
{\tt Lx}. JVM and CLR both have bytecodes for making an arbitrary 
jump to another bytecode, {\em but only within the current method}.
The stack unwinding is thus not directly possible. It can be obtained,
though, by using the exception mechanism. 
At the point of {\tt L}, an exception handler is
set up to catch exceptions that may be thrown by {\tt leave} and/or
{\tt restart}.  {\tt Leave} or {\tt restart} are then implemented by
throwing the appropriate exceptions. The exception thrown has to
contain information about
\begin{enumerate}
\item The {\em} object to unwind to
\item Which {\em} label in this object is the target
\end{enumerate}
\noindent
At the label definition site the exception handler code must then test
whether the object to unwind to is identical to itself {\em and} the
target label is numbered identical to it's own label. If so, the
handler makes an unconditional jump to the first statement after the
labeled region. If both conditions are not met, the handler propegates
(re-throws) the exception further up the stack.

A special problem comes with implementing leave/restart across
Coroutine borders. As detailed below, the current Coroutine
implementation involves Threads, and the propagation of the
leave/restart exception would thus involve catching the exception in
the thread's entry point, and signal this to the parent thread. This
has not yet been implemented.

Although we have not yet made any performance measures of the
resulting code, we expect the use of the exception mechanism to
implement leave and restart to be inefficient.

\subsection{Pattern variables}
BETA supports patterns as first class
values that can be passed as arguments to other patterns and
returned as values. Special methods are generated to handle pattern
variables - space does not allow the details to be described here.
  

\section{Coroutines, concurrency and synchronization}
BETA has semi-coroutines in the style of Simula. BETA coroutines may be
executed in the usual non-premptive way as known form Simula but
also in concurrency with preemptive scheduling.  For synchronization
of concurrent coroutines, a {\tt Semaphore} pattern is avaliable to
express synchronization at the basic level. The BETA mechanism for
coroutines, concurrency and synchronization are at a much more
primitive level than in CJ. However, the powerful abstraction
mechanisms of BETA makes it easy to build higher-level concurrency
mechanisms such as monitors and Ada-like rendezvous and the BETA
libraries include a number of such concurrency abstractions.
  
The basic coroutines can be implemented in CJ using Threads, but this
requires considerable overhead -- basically a coroutine shift is much
simpler than a thread shift. The coroutine implementation in CLR/JVM
is probably the part of BETA that does not map very well. Space does
not allow to give the details of the coroutine implementation, which
is heavily inspired by \cite{helsgaun}.

\subsection{Quality of the mapping and changes to BETA}
Given the generality of BETA compared to Java, we find that
the current mapping to Java is satisfactory from a usability
point-of-view. The main nuisance is that two methods and an inner
class is generated for each pattern attribute. 

When the BETA project was initiated the main goal of unifying
abstraction mechanisms like class and method into patterns was to
obtain a uniform treatment of all abstraction mechanisms of the
language. Given a pattern abstraction it was expected that specialized
patterns for class and methods should be available.  This was termed
{\it language restriction}. However, in practice there was no real
demand/need for these special purpose abstractions. 

In the context of mapping BETA to Java, the idea of language
restrictions have been reintroduced. It is now in fact possible to
annotate a pattern with either {\tt class} or {\tt proc} meaning that
the pattern can only be used as a class or method. For such annotated
patterns, only one method needs to be generated. And for {\tt
  proc}-patterns the inner class may in fact be inlined -- the latter
optimization has not been implemented.

At this stage of the project, efficiency and optimization of the
mapping has had little attention.The first impression, however, is
that the JVM/CLR implementation results in code that is significantly
slower than the native BETA compilers. We do expect to able to improve
on parts of the implementation, but certain parts such as
leave/restart and coroutines the mapping appear to be inherently
inefficient.


\section{Conclusion}

\subsection{Evaluation}

\subsection{Contributions} 

\subsection{Lessons Learned}

\subsection{Platform issues}
\begin{itemize}
\item
No static link on stack frames
\item
Nested procedures/methods cannot be implemented using the stack
\item
Very rigid typing of class-fields 
\item
And method-locals on .NET
\item
Constructor initialization 
\item
Impossible to make setups before calling super constructor
\item
General exit out of nested method calls
\item
No support for covariant arguments
\item
No support for covariant return types
\item
Active objects
\item

.NET class references must be fully qualified, including location of binding
\item
Problems with separate compilation
\item
Large number of classes generated due to the generality of BETA patterns
\item
Java class files can only contain one (non-static) class per file 

\item
Lots of class files are generated
\item
.NET assemblies can contain any number of classes 
\item
No swap and dup\_x1/x2 on .NET
\item
Requirements for fully typed local variables in .NET. 
\item
Means that swap cannot be implemented in backend using local variables unless type of two top-of-stack objects are known.
\item
Type of field: super/this class

\end{itemize}

\subsection{Platform advantages}
Well-defined run-time format

More than just a calling convention for procedures

Memory management

Storage allocation

Object format \& method activation format

Garbage collection

Efficient code generation

We will have BETA for all Java- and .NET platforms

Can this be utilized efficiently?

If simple and direct mapping - yes

If complex mapping - no?

Inlining may help

Can rely on generating methods that the JIT inlines

\subsection{Language interoperability}
\note{Elaborate}

Java/\CS\  class inherited from BETA

Straight forward

Since name \& type-based,
just specify the methods you will use

Simpler than for COM --
COM uses offsets

All preceding methods must be declared

BETA pattern inherited from Java/\CS\  class --
Also straight forward

Issues with default constructor 

BETA object requires surrounding object (origin)

Issues --
Libraries and frameworks in Java/\CS\  -- 
Java-string, \CS-String, BETA-text

Automatic coercion implemented

Overloading

Constructors

External class interface

Interface syntax needed -- currently clumsy

External name \& location

Automatic include of interface files

.NET assemblies

Java class-files


\note{A summary of the experiences with language interoperability on
  JVM/CLR}


\begin{thebibliography}{sms}
\small
\bibitem{eclipsepaper}{Andersen, P., Enevoldsen, M.B., Madsen, O.L:
    Integration of BETA with Eclipse -- an exercise in language
    interoperability, submitted for publication}

\bibitem{helsgaun}{K. Helsgaun: Discrete Event Simulation in Java, 
Writings in Computer Science, Roskilde University, Denmark, 2000, 
{\tt http://www.dat.ruc.dk/~keld/research/JAVASIMULATION/
     JAVASIMULATION-1.0/docs/Report.pdf}}

\bibitem{lindskov}{J.L. Knudsen: A Static Exception Handling Mechanism? ...}


\bibitem{vclass}{O.L. Madsen, B. M�ller-Pedersen: 
Virtual Classes - A Powerful Mechanism in Object-Oriented Programming
In: Proceedings of OOPSLA'89, Object-Oriented Programming Systems, 
Languages and Applications, Sigplan Notices, 1989 }

\bibitem{betabook}{O.L. Madsen, B. M�ller-Pedersen, K. Nygaard:
    Object-Oriented Programming in the BETA Language. ACM
    Press/Addison Wesley, 1993. Out of print -- a copy can be downloaded
    from {\tt www.mjolner.dk}}\note{Here we should refer the new home
    page of BETA}

\bibitem{krab}{K. K. Thorup: Virtual classes in Java, ECOOP'97,
  Jyvaskala, 1997.}
  
\end{thebibliography}

\end{document}

