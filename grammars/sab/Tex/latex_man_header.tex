 
%The following definitions give an approximation of the style used in the
%LaTeX book:

% usage:  
%The following definitions give an approximation of the style used in the
%LaTeX book:

% usage:  
%The following definitions give an approximation of the style used in the
%LaTeX book:

% usage:  
%The following definitions give an approximation of the style used in the
%LaTeX book:

% usage: \input{latex_man_header}

\makeatletter
\input{fancyheadings.sty}
\addtolength{\headheight}{3pt}
\newcommand{\chaptermark}{}
% (Not def'ed in article.sty, in general it should be checked whether it is
% defined or not) 

\pagestyle{fancyplain}
% Make headwidth include marginal notes:
%\addtolength{\headwidth}{\marginparsep}
%\addtolength{\headwidth}{\marginparwidth}

% Ensure that the text of the head is changed when there is a new \section or
% \chapter. \chaptermark is called by \chapter, and sectionmark by \section
% (see latex.tex).
% Use \markboth to set left mark since there is no \markleft:
\renewcommand{\chaptermark}[1]{\markboth{#1}{#1}} 
\renewcommand{\sectionmark}[1]{\markright{\thesection\ #1}}
                                                
\lhead[\fancyplain{}{\bf\thepage}]{\fancyplain{}{\bf\rightmark}}
\rhead[\fancyplain{}{\bf\leftmark}]{\fancyplain{}{\bf\thepage}}
\cfoot{}
% (\rightmark gives value of rigth mark, set by \markrigth or \markboth,
% \leftmark gives value of left mark, set by \markboth)
\makeatother


\makeatletter
\input{fancyheadings.sty}
\addtolength{\headheight}{3pt}
\newcommand{\chaptermark}{}
% (Not def'ed in article.sty, in general it should be checked whether it is
% defined or not) 

\pagestyle{fancyplain}
% Make headwidth include marginal notes:
%\addtolength{\headwidth}{\marginparsep}
%\addtolength{\headwidth}{\marginparwidth}

% Ensure that the text of the head is changed when there is a new \section or
% \chapter. \chaptermark is called by \chapter, and sectionmark by \section
% (see latex.tex).
% Use \markboth to set left mark since there is no \markleft:
\renewcommand{\chaptermark}[1]{\markboth{#1}{#1}} 
\renewcommand{\sectionmark}[1]{\markright{\thesection\ #1}}
                                                
\lhead[\fancyplain{}{\bf\thepage}]{\fancyplain{}{\bf\rightmark}}
\rhead[\fancyplain{}{\bf\leftmark}]{\fancyplain{}{\bf\thepage}}
\cfoot{}
% (\rightmark gives value of rigth mark, set by \markrigth or \markboth,
% \leftmark gives value of left mark, set by \markboth)
\makeatother


\makeatletter
\input{fancyheadings.sty}
\addtolength{\headheight}{3pt}
\newcommand{\chaptermark}{}
% (Not def'ed in article.sty, in general it should be checked whether it is
% defined or not) 

\pagestyle{fancyplain}
% Make headwidth include marginal notes:
%\addtolength{\headwidth}{\marginparsep}
%\addtolength{\headwidth}{\marginparwidth}

% Ensure that the text of the head is changed when there is a new \section or
% \chapter. \chaptermark is called by \chapter, and sectionmark by \section
% (see latex.tex).
% Use \markboth to set left mark since there is no \markleft:
\renewcommand{\chaptermark}[1]{\markboth{#1}{#1}} 
\renewcommand{\sectionmark}[1]{\markright{\thesection\ #1}}
                                                
\lhead[\fancyplain{}{\bf\thepage}]{\fancyplain{}{\bf\rightmark}}
\rhead[\fancyplain{}{\bf\leftmark}]{\fancyplain{}{\bf\thepage}}
\cfoot{}
% (\rightmark gives value of rigth mark, set by \markrigth or \markboth,
% \leftmark gives value of left mark, set by \markboth)
\makeatother


\makeatletter
\input{fancyheadings.sty}
\addtolength{\headheight}{3pt}
\newcommand{\chaptermark}{}
% (Not def'ed in article.sty, in general it should be checked whether it is
% defined or not) 

\pagestyle{fancyplain}
% Make headwidth include marginal notes:
%\addtolength{\headwidth}{\marginparsep}
%\addtolength{\headwidth}{\marginparwidth}

% Ensure that the text of the head is changed when there is a new \section or
% \chapter. \chaptermark is called by \chapter, and sectionmark by \section
% (see latex.tex).
% Use \markboth to set left mark since there is no \markleft:
\renewcommand{\chaptermark}[1]{\markboth{#1}{#1}} 
\renewcommand{\sectionmark}[1]{\markright{\thesection\ #1}}
                                                
\lhead[\fancyplain{}{\bf\thepage}]{\fancyplain{}{\bf\rightmark}}
\rhead[\fancyplain{}{\bf\leftmark}]{\fancyplain{}{\bf\thepage}}
\cfoot{}
% (\rightmark gives value of rigth mark, set by \markrigth or \markboth,
% \leftmark gives value of left mark, set by \markboth)
\makeatother
