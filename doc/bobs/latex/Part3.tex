
\part{} 
\section{Program Description}
In this section the program organization is outlined and algorithms
and data structures of special interest are described in detail.

The description is not of any interest to the ordinary user
of the system and may be skipped.
No important information about the normal use of the system is
contained in this part.

The present description is addressed to users (or just readers) 
who are specially interested in a detailed description of:
\begin{itemize}
\item the program organization
\item the methods behind the algorithms, or
\item the actual appearance of the algorithms.
\end{itemize}

The reader is assumed to be familiar with the terminology, and
the concepts in consideration are assumed to be well known to the
reader. References [1, 2, 3, 4, 5, 18, 19, 20] may be helpful to
provide the necessary background.

The description consists of two parts:
\begin{itemize}
\item a description of the parser--generator,
\item a description of the Skeleton=compiler.
\end{itemize}

\subsection{Description of the Parser--Generator}
The parser--generator program may be described as three separate 
modules:
\begin{itemize}
\item the input module, which transforms an external grammar 
specification into an internal representation,
\item the grammar transformation module, which consists of a 
set of routines, each performing either a grammar check or a 
grammar transformation - or both,
\item the parse table construction module which builds the parse
tables according to the specified grammar. 
The parse tables are organized as a directed cyclic graph where 
the nodes are linked lists.
The information part of a given list is either a sequence of 
grammar symbols or a sequence of references to graph nodes.
\end{itemize}

\subsubsection{The Construction of the Parse Tables}
The construction of the parse tables consists of a sequence of steps:
\begin{itemize}
\item The construction of the LR(0) tables; (which is a realization
of the method described in [2]).
\item The extension of LALR(1)-tables which supplies any 
inadequate table among the LR(0)-tables with LALR(1)-lookahead
information, which may solve the existing conflict.
A table supplied in this way is called a lookahead table.
SL(1) lookahead information may be sufficient to solve a conflict 
and may then be used.
\item The construction of special lookback tables, defined in [1].
The tables are constructed on the basis of the grammar and the LR(0) tables:
one lookback table for each nonterminal symbol. 
This construction method deviates from the one given in [1] 
but includes automatically most of the optimizations of the lookback
tables, also given in [1].
\item The optimization of the lookback tables, described in [1, 4],
which involves the addition of an ``In Any Case'' condition for the
most popular destination table in a lookback table;
\item the optimization of the parse tables which removes the 
information concerning the transitions on nonterminal symbols, given
in [1]. When the lookback tables are added to the parse tables this
information becomes redundant.
\item The optimization of the lookahead tables, first described in
[4], which involves the addition of an ``In Any Case'' condition for
the most popular destination table in a lookahead table.
\item The optimization of (unaltered) LR(0) tables which merges 
tables having identical bottom-most subsets.
The tables are modified to share the common part.
This optimization may be regarded as a variant of the method 
using top-most subsets, given in [4].
\end{itemize}

\subsubsection{The LALR(1) Lookahead Algorithm}
A description of the LALR(1) algorithm may be found in [23].

\subsection{Description of the Skeleton Compiler}
This program consists logically of two parts:
\begin{itemize}
\item {\em The semantic interface}
which consists of a list of predefined structures and auxiliary 
routines which may be useful to the user.
Furthermore these define the means by which the user and the
parser may interact, (see section 5.2).
\item {\em The parser}
which primarily consists of a parsing algorithm working on the
parse tables constructed by the parser generator program.
This parsing algorithm is described in the next section.

The lexical analyzer is an independent part of the parser.
It runs on tables which are also built up by the parser generator 
program.

The parser includes an error recovery routine based on the parse 
tables. 
This routine is described in detail in section 6.2.2.
\end{itemize}

\subsubsection{The Parsing Algorithm}
In addition to the parse tables, the parsing algorithm uses a 
parse stack.
A {\em picture} is a special parse situation which we 
may define as follows:

Assume that the parse tables are constructed on the basis of a
$CFG, G = (N, \Sigma, P,S)$. Let $t_k$ be a parse table,
$k \in [0,i])$, and $X_k \in N \cup \Sigma, (k
\in [1,i-1])$, and $X_i = a_j$, and $a_k \in
\Sigma, (k \in [1,n])$ and $1 \leq j \leq n$.


The {\bf general  picture}, $P_j$, is then:
\begin{tabbing}
\hspace{1cm} \= \hspace{6cm}\= \kill
\> \ramme[5cm][l]{$t_0 X_1 t_1 X_2 t_2 \ldots X_i t_i$} 
\> $a_{j+1} a_{j+2} \ldots a_n$\\
\>\hspace{1.5cm}parse stack
\end{tabbing}

The {\bf initial picture},$P_0$, is:

\begin{tabbing} 
\hspace{1cm} \= \hspace{6cm} \= \kill
\> \ramme[5cm][l]{$t_0$} \>  $a_1 a_2 \ldots a_n$ 
\end{tabbing}

and the {\bf final picture}, $P_n$, is:

\hspace*{1cm}
\ramme[5cm][l]{$t_0 S$}

The picture $P_j$, $0 < j$, denotes the situation where
the symbol $a_j$ (and a parse table $t_i$) has just been shifted
into the stack.
The parse tables $t_{i-1}$ and $t_i$, which are also on the stack
mean that there exists a transition path on the symbol $a_j$ from
table $t_{i-1}$ to table $t_i$. 
This fact may be generalized to any sequence on the stack of the
form $t_{k-1} X_k t_k$.

Assuming that the string $a_1 a_2 \ldots a_n$ is in the L(G)
the parsing algorithm runs through the sequence of pictures

\[
P_{0}, \;P_{1}, \;\ldots,\; P_n.\]

Between any two pictures in this sequence the stack may increase
and decrease according to intermediate parse situations.
The parsing algorithm may be specified as:

\begin{tabbing}
\hspace{1 cm} \= BEGI \= \kill
\> $P := P_0$;\\
\> WHILE input not exhausted DO\\
\> BEGIN\\
\> \> IF $\neg$ lookahead $(P)$ THEN recover $(P)$;\\
\>�\> $P$ := successor $(P)$;\\
\> END; 
\end{tabbing}

The routines involved behave as follows:

-- Lookahead,\\
determines if there exists a picture $P_{i+1}$ as a successor to
$P_i$.
The calculation process may involve a sequence of both reductions
of the parse stack and shifts onto the parse stack.
In this context a shift is due to a reduction of an empty
production.
$P_i$ is not touched if the stack is extended into an area on top of
the original stack.
The differences between the original picture and the succeeding
picture are contained in this area.

The picture $P_1$ may be depicted as follows:
\vspace*{3cm}


The three indices into the stack all indicate the actual top of the
stack.
In the lookahead routine the stack may change as described above.
Thus the stack situation during the computation may appear as:


\vspace*{3cm}

As stated the original picture $P_i$ must be protected.
The area in question is indicated by the index ORIGINAL, which
remains fixed during the lookahead computation.

The hatched area of the above illustration shows the part of the
stack which is being used in the computations at this moment. 
The area situated between ORIGINAL and PSEUDO contains a part of
stack which might have been placed in the unhatched area upwards
from VALID (thus causing the destruction of the original stack).
The special case (which in fact is the common case) when this area
between ORIGINAL and PSEUDO is empty, is perceived according to the
following illustration:

\vspace*{3cm}


The area (in both the latter figures) below VALID is the part of
the original stack which is still in use at the moment.

We note that the index PSEUDO may both increase and decrease,
whereas VALID may only decrease.

When the lookahead routine terminates it leaves a stack situation
similar to one of the illustrations.

During the lookahead computation adequate information about the
possible sequence of reductions (mentioned above) is queued up for
use in the successor routine.

-- Successor,\\
deliveres the information about the possible reduction sequence to
the semantic interface part of the program.

On the basis of the extra information in the area on the top of the
parse stack a succeeding stack is built, resulting in the picture
$P_{i+1}$.

The transformation of the intermediate stack to the succeeding
picture may be depicted:

\vspace*{4cm}


(The transformation of the alternative intermediate stack situation
is trivial.)
All three indices are reset to the resulting top of the stack.

- Recover,\\
transforms the picture $P_i$ into another picture $P'_k$, such 
that Lookahead $(P'_k)$ is TRUE. Recover involves the 
construction of another stack and the deletion of a number
 of input symbols. The routine is described in detail in 
the next section.

A somewhat different description of the parsing principle appiled
above is given in [19].

\subsubsection{The Recovery Algorithm}
The recovery algorithm is invoked when parsing a string not in L(G).
However, having detected an error, several possibilities of reaction
exists.
A safe but normally insufficient reaction is to stop the parsing
and inform the user about the discovered error.
A tempting reaction is to try to correct the input string, i.e. \ to
transform the erroneous input string into an error--free string.
However, in general the correction scheme is extremely complicated.
An appealing compromise is the recovery principle in which the
parser establishes a situation such that a major part of the
remaining input string may also be inspected to discover any
additional errors.
The process may involve both the restructuring of the parse stack
and the deletion of symbols from the input string.

{\em 6.2.2.1 Description of the Algorithm}\\

The recovery algorithm is activated from the parsing algorithm when
lookahead $(P_j)$ is FALSE for some $j$. Let $P_j$ be\\
\begin{tabbing}
\hspace{1cm} \= \hspace{6cm} \= \kill
\> \ramme[5cm][l]{$t_0 X_1 \ldots X_i t_i$} \> 
$a_{j+1}a_{j+2} \ldots a_n$
\end{tabbing}

The recovery method compiles with an outline given in [18] and
works by applying error productions.
An error production has the form $A \rightarrow \alpha\;
\mbox{error}\; \beta$ where error is a special symbol.

In the algorithm the remaining input symbols are inspected one
symbol at a time.
For each symbol a sequence of stack situations is inspected, which
is constructed from the original stack by popping one stack element
at a time.
For each such stack situation the special error symbol is inserted
as the first symbol on input.
This artificial picture is now inspected.
If a successor to this picture exists, to which, furthermore,
another successor exists, then a straightforward way of recovering
from the error situation has appeared.
(Involving the described deletion of input symbols and popping of
stack elements.)

The method may be formalized in the following algorithm: (based on
$P_{j})$

\begin{tabbing}
FORi \= BEG \= BEG \= BEG \= \kill

FOR $k := j + 1$ TO $n$ DO\\
\> FOR $I := i$ DOWNTO 0 DO\\
\> BEGIN\\
\> \> LET $P^{l,k}_{error}$\\
\> \> BE \ramme[3.5cm][l]{$t_0 X_1t_1\ldots X_lt_l$}  error
$a_k \ldots a_n$\\
\> \> IF lookahead $(P^{l,k}_{error})$ THEN\\
\> \> BEGIN\\
\> \> \> $P'_k$ := successor $(P^{l,k}_{error})$;\\
(*) \> \> \> IF lookahead $(P'_k)$ THEN\\
\> \> \> BEGIN\\
\> \> \> \> delete symbols $a_j, a_{j+1},\ldots, a_{k-1}$;\\
\> \> \> \> EXIT TO success;\\
\> \> \> END;\\
\> \> END;\\
\> END;\\
\> success: (* successful picture: $P'_k$ *)
\end{tabbing}

Note: A straightforward improvement of the algorithm is to extend
the check performed at ($^*$) to include 2-3 symbols instead of only a
single symbol.

{\em 6.2.2.2 Description of the Recovery Principle}\\

The detection of an error is caused by a picture $P_j$:
\ramme[2.5cm][l]{$t_0 \ldots X_i t_i$} $a_{j+1} \ldots
a_n$, where $X_i = a_j$.

The existence of $P_j$ implies $\exists y \in \Sigma^*$ such
that
\begin{eqnarray}
 S & \Rightarrow^*_{rm} & X_1 \ldots X_i y\\
& \Rightarrow^*_{rm}& a_1 \dots a_j y \nonumber
\end{eqnarray}

The recovery method involves a specification of $k$ and $l$ $(j
< k \leq n$,$0 \leq l \leq i)$ and thereby the picture

$P^{l,k}_{error}: t_0 \ldots X_l t_l\; \mbox{error}\;
 a_k \ldots a_n$ with the property that
 

\begin{eqnarray}
\mbox{lookahead} (P^{l,k}_{error}) &=&TRUE
\end{eqnarray}

and assuming that

\begin{eqnarray*}
P^l_k := \;\mbox{successor}\; (P^{l,k}_{\mbox{error}}) 
\end{eqnarray*}

the property that


\begin{eqnarray}
\mbox{lookahead}\; (P^l_k) &= &TRUE.
\end{eqnarray}

Inspecting the method step by a sequence of conclusions may be
stated based on the results above:

\begin{enumerate}
\item implies that for a given $l \leq i, \exists s (0 \leq s 
\leq j)$ and $\exists z \in \Sigma^*$ such that
  
    \begin{tabbing}
xxxxx \= $S$n \= $\Rightarrow^{*}_{rm}$ \=  \kill
 \>   $S$ \>  $\Rightarrow^{*}_{rm}$ \>  $X_1 \ldots X_l X_{l+1} 
            \ldots X_i y$\\
 \> \>   $\Rightarrow^{*}_{rm}$ \>  $X_1 \ldots  X_i zy$\\
 \> \>   $\Rightarrow^{*}_{rm}$ \>  $a_1 \ldots  a_s zy$
    \end{tabbing}

\item may be interpreted to mean that there exist $x =
x'' x' \in \Sigma^*, m \;(0 \leq m \leq l)$ and an error
production $A \rightarrow \alpha$  error  $\beta$ such that

        \begin{tabbing}
xxxxx \= $S$n \= $\Rightarrow^{*}_{rm}$ \= \kill
\>    $S$ \> $\Rightarrow^*_{rm}$ \> $X_1 \ldots X_m A x'$\\
 \> \>   $\Rightarrow^*_{rm} $\> $X_1 \ldots X_m
    X_{m+1} \ldots X_l\; \mbox{error}\; \beta
    x'$\\ 
\> \> $\Rightarrow^*_{rm}$ \> $X_1 \ldots X_l\;
    \mbox{error}\; x'' x'$\\
\> \> $\Rightarrow^*_{rm}$ \> $a_1 \ldots a_s \;\mbox{error} \;x$
   \end{tabbing}

where $\beta \Rightarrow^*_{rm} x''\; \mbox{and}\; \alpha =
X_{m+1} \ldots X_l$.

\item implies that at least one of the potential error
productions, say $A \rightarrow \beta$, is legal when the symbol
$a_k$ is included. Hence $\exists w = w''\, w' = a_k v$ such that

           \begin{tabbing}
xxxxx \= $S$\= $\Rightarrow^{*}_{rm}$\=  \kill
 \> $S$ \>  $\Rightarrow^*_{rm}$\> $X_1 \ldots X_m A w'$\\
 \> \>  $\Rightarrow^*_{rm}$ \>  $X_1 \ldots X_1\; \mbox{error}\;
     \beta w'$\\
 \> \> $\Rightarrow^*_{rm}$ \>  $X_1 \ldots X_1
     \mbox{error}\, w'' w'$\\
 \> \> $\Rightarrow^*_{rm}$ \> $a_1 \ldots a_2 \;\mbox{error}\;
     a_k v$
     \end{tabbing}

where $\beta \Rightarrow^{*}_{rm} w''\;\mbox{and}\; \alpha =
X_{m+1} \ldots X_l.$
\end{enumerate}

We may now summarize the results.
The parsing algorithm has gone through the parse situations

\begin{tabbing}
\hspace{1cm} \= \hspace{7cm}\= \kill
\> \ramme[5cm][l]{$t_0 X_1 t_l \ldots t_l X_1$} \> $a_{s+1} \ldots 
a_j�\ldots a_k \ldots a_n$\\
\mbox{}\\
and\\
\mbox{}\\
\>\ramme[5cm][l]{$t_0 X_1 t_1 \ldots t_lX_l\ldots
 t_i X_i$} \> $a_{j+1} \ldots a_k \ldots a_n$\\
\mbox{}\\
at which the error is detected.\\
\mbox{}\\
This situation is transformed into\\
\mbox{}\\
\>\ramme[5cm][l]{$t_0 X_1 t_1 \ldots t_lX_l$}
\> error $a_k \ldots a_n$
\end{tabbing}

The effect of the recovery method is to discard the symbols
$X_{l+1}\ldots X_i$ from the stack and to delete the
symbols $a_{j+1}\ldots a_{k-1}$ from the input string.
Recalling that

\hspace*{1cm}
$X_{l+1} \ldots X_i\; \Rightarrow^*_{rm}\; a_{s+1} \ldots a_j$

the total effect is to discard the symbol string $a_{s+1} \ldots
a_{k-1}$.
The string is replaced by the special symbol error which is part of
a string derivable from the language construct $A$.

Now the philosophy behind the recovery method may be expressed as:
While parsing the string $a_1 \ldots a_n$ an error is detected.
The error is located in the partial string $a_{s+1} \ldots
a_{k-1}$ which is supposed to reduce to a part of the language
construct $A$. 
A part of the string has been reduced into
$X_{l+1} \ldots X_i$ and a part remains on input. 
Both
these parts are discarded.
The succeeding picture $P'_k$ : \ramme[2,5cm][l]{$t_0 \ldots
X_r t_r$} $\;\;\; a_k \ldots a_n\;$ may be interpreted as an
element of the picture sequence evaluated from the initial picture

\hspace*{1cm}$P'_0:$ \ramme[1cm][l]{$t_0$} $\;\;\; 
a_1 \ldots a_s\; \mbox{error}\;
a_k \ldots a_n$.

Pictures among the successors of the picture $P'_k$ may cause new
error detections.

